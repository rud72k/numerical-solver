\documentclass{article}
    \usepackage{amsmath,amssymb,amsfonts,amsthm}
    \usepackage[margin=1.2in]{geometry}
    \usepackage{setspace}
    \usepackage{xcolor}
    \usepackage[left, mathlines]{lineno}
    \usepackage{hyperref}
    \usepackage{color}


    \newcommand{\dx}{\text{ d}x}
    \newcommand{\dt}{\text{ d}t}
    \newcommand{\half}{\frac{1}{2}}
    \newcommand{\dhdt}{\frac{\partial h}{\partial t}}
    \newcommand{\dhdx}{\frac{\partial h}{\partial x}}
    \newcommand{\dudt}{\frac{\partial u}{\partial t}}
    \newcommand{\dudx}{\frac{\partial u}{\partial x}}



    \theoremstyle{definition}
    \newtheorem{definition}{Definition}[section]
    \setstretch{1.5}
    \title{Analysis of the SBP-SAT Stabilization with Finite Volume: Advection Equation}
    \author{name}
    \linenumbers
\begin{document}
\maketitle
% \section{Introduction}

\section{SBP-SAT Approximation}
    Consider the following scalar advection equation in 1D in the domain $x \in [0,1]$. 
    \begin{equation}\label{adv}
        \frac{\partial q(x,t)}{\partial t} + c(x)\frac{\partial q(x,t)}{\partial x} = 0 
    \end{equation}

    Here, $q(x,t)$ be the quantity of the water advected along the domain 
    and $c(x)$ be the velocity which it advected. 
    Let the initial condition in that domain be $q(x,0)=f(x)$. 
    In this case, we let the water goes in from the left following the equation 
    $q(0,t)=g(t)$ and advected throughout the domain with 
    speed $c(x)=1$ and let out at $q(1,t)$. 
    Such a problem called Initial Boundary Value Problem (IBVP).
    We want to solve the equation numerically with semidiscrete 
    method which energy stable. 
    

\subsection{Stability of The Model}
    Introducing the energy method by define the inner product on $L_2$ as well as the norm.
    For any smooth function $p$ and $q$, define the inner product as 
    \begin{align} (p,q) = \int_0^1 p(x)q(x)\ \text{ d}x \end{align}
    and the $L_2$-norm as 
    \begin{align} \Vert q(\cdot,t) \Vert ^2 = (q,q) \end{align}
    Later on, we want to mimic this energy method and redefine it in the semidiscrete
    method to prove the stability. 

    Before that, we will see how the boundary condition affects stability of our advection 
    equation. 
    
    Now consider the advection equation (\ref{adv}) with initial condition 
    $q(x,0)=f(x)$ and boundary condition $q(0,t)=g(t)$. 

    Now, multiplied by the solution $q$. We have $(q,q_t + q_x) = 0$ or equivalently
    \begin{align}
        (q,q_t) &+ (q,q_x) = 0 \\
        \int_0^1 q(x,t)\frac{\partial q(x,t)}{\partial t} \text{ d}x &+ 
        \int_0^1 q(x,t)\frac{\partial q(x,t)}{\partial x} \text{ d}x = 0
    \end{align}
    With the same argument, we will have
    \begin{align}
        (q_t,q) + (q_x,q) = 
        \int_0^1 \frac{\partial q(x,t)}{\partial t} q(x,t)\text{ d}x+ 
        \int_0^1 \frac{\partial q(x,t)}{\partial x} q(x,t) \text{ d}x &= 0
    \end{align}
    Adding the two, we will have 
    \begin{align}    
        \int_0^1 \left(q(x,t)\frac{\partial q(x,t)}{\partial t} + 
        \frac{\partial q(x,t)}{\partial t} q(x,t) \right) \text{ d}x 
        + 2\int_0^1 \frac{\partial q(x,t)}{\partial x} q(x,t) \text{ d}x = 0
    \end{align}
    

    Integrated by part on the first integral gives us 
    \begin{align}
        \int_0^1 \left(q(x,t)\frac{\partial q(x,t)}{\partial t} + 
        \frac{\partial q(x,t)}{\partial t} q(x,t) \right) \text{ d}x  
        = \int_0^1 \frac{\partial q(x,t)^2}{\partial t} \text{ d}x 
        = \frac{\partial}{\partial t} \int_0^1 q(x,t)^2 \text{ d}x 
        = \frac{\partial}{\partial t} \Vert q(x,t)\Vert.  
    \end{align}
    Meanwhile, we also have 
    \begin{align}
        \int_0^1 2\frac{\partial q(x,t)}{\partial x} q(x,t) \text{ d}x 
        = \int_0^1 2\ q(x,t) \partial q(x,t) = q(x,t)^2\mid_0^1 
        = q(1,t)^2 - q(0,t)^2
    \end{align}

    Now we have 
    \begin{align}
        \frac{\partial}{\partial t} \int_0^1 q(x,t)^2 \text{ d}x 
        = - q(1,t)^2 + q(0,t)^2 = - q(1,t)^2 + g(t)^2
    \end{align}

    Thus, $\frac{\partial}{\partial t} \Vert u(\cdot,t) \Vert 
    =   - q(1,t)^2 + g(t)^2$. 
    Since $q(1,t)^2$ always positive, then we could see that 
    the energy bounded above by $g(t)^2$, so it is stable. 

\subsection{Semidiscrete Model}
    Introduce integration by part property with the product notation. 
    Noting that 
    \begin{align}
        ((\partial p/\partial x),q) = \int_0^1 \frac{\partial p}{\partial x}q \dx 
        &= -\int_0^1 \frac{\partial q}{\partial x}p \dx + p(0)q(0) - p(1)q(1) \\
        &= - (p,(\partial q/\partial x)) + p(1)q(1) - p(0)q(0)
    \end{align}
    or equivalently 
    \begin{align}
        ((\partial p/\partial x),q) + (p,(\partial q/\partial x) = p(1)q(1) - p(0)q(0)
    \end{align}
    \begin{definition}\label{def-sbp-operator}
        Let $N$ be a natural number and $\mathbf{x} = [x_1,x_2,\dots,x_N]$ be a discretization of 
        the interval that $p$ and $q$ are defined. Let $\mathbf{p}$ and $\mathbf{q}$ are the 
        discretization of $p$ and $q$ on $\mathbf{x}$ respectively. A discrete operator ${D}$ said to be 
        have summation by part properties if there exist discrete inner product 
        $<\cdot,\cdot>_H$ along with the symmetric matrix $H$ such that 
        \begin{align}
            <D\mathbf{p},\mathbf{q}> + <\mathbf{p}, D \mathbf{q}> &= \mathbf{p}(x_N)\mathbf{q}(x_N) 
                - \mathbf{p}(x_0)\mathbf{q}(x_0)
            % <D_{xx} u, v> &= - u^\top M^{(a)}v + (\bar{{q}}_N a_n (SV)_n - \bar{{q}}_0 a_0 (SV)_0)
        \end{align}
        with %$SV \approx \frac{\partial p}{\partial x}$ and 
        $H$ be 
        positive semi definite matrices. % need to defined what p(0) is

        % relation with equation (13)
    \end{definition}

    In order to approximate the time derivative numerically, we will use derivative operator 
    in the form $D = H^{-1}Q$ with $D$ being operator that derived from Finite Volume Method. 

    Consider more general equation than (\ref{adv})
    \begin{equation}\label{adv2}
        \frac{\partial q}{\partial t} + \frac{\partial f}{\partial x} = 0
    \end{equation}
    defined on the interval $[0,1]$ for some operator $f$ with initial and boundary condition 
    $q(x,0)=k(x)$ and $q(0,t)=g(t)$. 
    for some operator $f$. 
    
    Consider Finite Volume Method on identical grid on the interval $[0,1]$. 
    Before continuing, we'll discretize the interval into some grids. 
    Let $N$ be a positive integer and $I_i$ be the $i$-th control volume for all $i=1,2,...,N.$.
    Define $\Delta x_i = x_{i+\half} - x_{i-\half}$ be the length of the $i$-th interval 
    and $x_{i-\frac12}$ and $x_{i+\frac12}$ are the left boundary and right boundary
    of the interval respectively. 

    From (\ref{adv2}),
    we have 
    \begin{align}
        \int_{I_i} \frac{\partial q}{\partial t}(x,t)\text{ d}x 
            &+  \int_{I_i} \frac{\partial f}{\partial x}(x,t)\text{ d}x = 0\\ 
        \dfrac{d}{dt}\int_{I_i} q(x,t) \text{ d}x
            &+ \left( f(x_{i+\frac12},t) - f(x_{i-\frac12},t) \right)= 0
    \end{align}
    Now, introducing the following notation.
    \begin{align}
        \bar{{q}}_i = \frac{1}{\Delta x_i}\int_{I_i} q(x,t) \dx \\
        f_{i \pm \frac12} = f(x_{i\pm \frac12},t)
    \end{align}
    Using the new notation and rearrange terms, the last equation becomes
    \begin{equation}\label{sd1}
        {\Delta x_i}\frac{d\bar{{q}}_i}{dt} + \left( f_{i+\frac12} - f_{i-\frac12}  \right) = 0
    \end{equation}
    Now what's left is to approximate the fluxes $f_{i \pm \frac12}$ with numerical flux $F_{i \pm \frac12}$
    \begin{align}\label{sd2}
        {\Delta x_i}\frac{d\bar{{q}}_i}{dt} + \left( F_{i+\frac12} - F_{i-\frac12}  \right) = 0
    \end{align}
    The local flux by the inflow and outflow by the average flux on the neighbouring cell, %russanov, lax frederich
    and adding the dissipative energy, we would get 
    \begin{equation}\label{approx1}
        F_{i+\frac{1}{2}} = \frac{f(\bar{{q}}_{i+1}) + f(\bar{{q}}_i)}{2} 
        - \frac{\alpha}2\left(\bar{{q}}_{i+1}  - \bar{{q}}_i\right)
    \end{equation} 
    with some constant $\alpha >0$. 
    Calculating $\left( F_{i-\frac12} - F_{i+\frac12}  \right)$ we get
    \begin{align}
        \left( F_{i-\frac12} - F_{i+\frac12}  \right) 
                &= \left(\frac{f(\bar{{q}}_{i+1}) + f(\bar{{q}}_i)}{2} 
                - \frac{\alpha}2\left(\bar{{q}}_{i+1}  - \bar{{q}}_i\right) \right) 
                 - \left( \frac{f(\bar{{q}}_{i}) + f(\bar{{q}}_{i-1})}{2} 
                 - \frac{\alpha}2\left(\bar{{q}}_{i}  - \bar{{q}}_{i-1}\right) \right) \\
                &= \frac{f(\bar{{q}}_{i+1})- f(\bar{{q}}_{i-1})}{2} 
                - \frac{\alpha \left(\bar{{q}}_{i+1} - 2\bar{{q}}_i + \bar{{q}}_{i-1}\right)}2
    \end{align}
    Put this approximation into (\ref{sd2}), we got 
    \begin{align}\label{fluxmid1}
        \Delta x \frac{d\bar{{q}}_i}{dt} + \frac{f(\bar{{q}}_{i+1}) -f(\bar{{q}}_{i-1}) }{2} 
        - \frac{\alpha}2 {\left(\bar{{q}}_{i+1} - 2\bar{{q}}_i + \bar{{q}}_{i-1}\right)} = 0
    \end{align}

    \newpage
    Meanwhile, in the left and right boundary, 
    we don't have any cell on it's left and right, respectively. 
    So we could say that the length of the leftmost (rightmost) cell 
    is cut in a half. That means, instead of divide by $\Delta x$, 
    we divide by $\Delta x/2$. That is 
    \begin{align}
        \label{approx2}\frac{\Delta x}2 \frac{d\bar{q_1}}{dt} 
                            + \left( F_{\frac12} -F_{0}  \right) = 0 \\
        \label{approx3}\frac{\Delta x}2  \frac{d\bar{q_N}}{dt} 
                            + \left( F_{N} - F_{N-\frac12}  \right) = 0 
    \end{align}
    % F_0 = f(q_0)
    % F_N = f(q_N)
    The fact that we don't have whole cell on the left and the right force us to 
    set $F_0 = f(\bar{q}_0)$ and $F_N = f(\bar{q}_N)$
    and (\ref{approx1}) to (\ref{approx2})  we have
    \begin{align}
        \frac{\Delta x}2 \frac{d\bar{q_1}}{dt} + \left( F_{\frac12} -F_{0}  \right) 
                &= \frac{\Delta x}2 \frac{d\bar{q_1}}{dt} 
                + \left(\frac{\bar{q}_{1} +\bar{q}_{0}}{2} 
                - \frac{\alpha (\bar{q}_{1}-\bar{q}_{0}))}{2} \right) - \bar{q}_0 \\
                &=\frac{\Delta x}2 \frac{d\bar{q_1}}{dt} 
                + \frac{\bar{q}_{1} -\bar{q}_{0}}{2} 
                - \frac{\alpha (\bar{q}_{1}-\bar{q}_{0}))}{2}  
    \end{align}
    and with the same manner we also have the following. 
    \begin{align}
        \frac{\Delta x}2 \frac{d\bar{q_N}}{dt} + \left( F_{N} -F_{N-\frac12}  \right) 
        &= \frac{\Delta x}2 \frac{d\bar{q_N}}{dt} +\bar{q}_N 
        - \left(\frac{\bar{q}_{N} +\bar{q}_{N-1}}{2} 
        - \frac{\alpha (\bar{q}_{N}-\bar{q}_{N-1}))}{2} \right)\\
        &= \frac{\Delta x}2 \frac{d\bar{q_N}}{dt} 
        + \frac{\bar{q}_{N} - \bar{q}_{N-1}}{2} 
        + \frac{\alpha (\bar{q}_{N}-\bar{q}_{N-1}))}{2} 
    \end{align}
    Hence, in the boundary we'll have 
    \begin{align}
        \label{fluxleft}\frac{\Delta x}2 \frac{d\bar{{q}}_0}{dt} 
            + \frac{{f(\bar{{q}}_{1}) - f(\bar{{q}}_{0}) }}{2} 
            -  \frac{\alpha}2 {\left(\bar{{q}}_{0} - \bar{{q}}_1 \right)} &= 0\\
        \label{fluxright}\frac{\Delta x}2 \frac{d\bar{{q}}_N}{dt} 
        + \frac{f(\bar{{q}}_{N}) -f(\bar{{q}}_{N-1}) }{2} 
        + \frac{\alpha}2 {\left(\bar{{q}}_{N} - \bar{{q}}_{N-1} \right)} &=0 
    \end{align}
    Putting together equation (\ref{fluxleft}),(\ref{fluxmid1}), and (\ref{fluxright}) respectively, 
    we'll have the following system of equation. 
    \begin{align}
        \frac{\Delta x_0}{2} \frac{d\bar{{q}}_0}{dt} + \frac{f(\bar{{q}}_{1}) - f(\bar{{q}}_{0}) } 2
            &+  \frac{\alpha}2{\left(\bar{{q}}_{0} - \bar{{q}}_1 \right)}  &= 0 \ & \notag \\
        \Delta x_i \frac{d\bar{{q}}_i}{dt} + \frac{f(\bar{{q}}_{i+1}) -f(\bar{{q}}_{i-1}) }{2} 
            &+ \frac{\alpha}2 {-\left(\bar{{q}}_{i+1} + 2\bar{{q}}_i - \bar{{q}}_{i-1}\right)} &= 0, 
                &\textrm{ for }i=1,2,\dots,N-1 \notag \\
        \frac{\Delta x_N}2 \frac{d\bar{{q}}_N}{dt} 
            + \frac{f(\bar{{q}}_{N}) -f(\bar{{q}}_{N-1}) }2
            &+ \frac{\alpha}2 {\left(\bar{{q}}_{N} - \bar{{q}}_{N-1} \right)} &=0 \ \label{system1}&
    \end{align}

    % We could see easily see that both $A$ and $H^{-1}A$ are {symmetric}. Now, let  $D = H^{-1}Q$.   %$D = H^{-1}Q$. 
    % % With some abuse of notation, the partial derivative system (\ref{system1}) could be rewrite as 
    % \begin{align}
    %     H \frac{d\mathbf{q}}{dt} + Q \mathbf{f}(\mathbf{q}) + \alpha A(\mathbf{q}) = 0
    % \end{align}
    % or equivalently,
    % \begin{align}\label{system2}
    %     \frac{d\mathbf{q}}{dt} + D \mathbf{f}(\mathbf{q}) + \alpha H^{-1}A\mathbf{q} = 0
    % \end{align}


    % with $\mathbf{q} = (q_0, q_1, \dots, q_N)^\top $, 
    % and let $Q$ as follows: 
    % \begin{equation}
    %     Q = \begin{pmatrix}
    %             - \frac12         & \frac12            & 0              & 0               & \cdots & 0        & 0         & 0 \\
    %             -\frac{1}{2} & 0             & \frac{1}{2}    & 0               & \cdots & 0        & 0         & 0 \\
    %             0            & - \frac{1}{2} & 0              & \frac{1}{2}     & \cdots & 0        & 0         & 0 \\
    %             \vdots       & \vdots        & \vdots         & \vdots          & \ddots & \vdots   & \vdots    & \vdots \\
    %             0            & 0             & 0              & 0               & \cdots & -\frac{1}{2} & 0      & \frac{1}{2} \\
    %             0            & 0             & 0              & 0               & \cdots & 0        & -\frac12      & \frac12
    %     \end{pmatrix}
    % \end{equation}
    % and matrix $H =  \text{diag} (\frac{1}{2}\Delta x_0,\Delta x_1,\Delta x_2,\cdots,\Delta x_{N-1},\frac{1}{2}\Delta x_N)$
    % and
    % \begin{equation}
    %     A = \begin{pmatrix}
    %         {1}           &{-1}            & 0              & 0               & \cdots & 0        & 0         & 0 \\
    %         -1           & 2             & -1    & 0               & \cdots & 0        & 0         & 0 \\
    %         0            & -1            & 2              & -1      & \cdots & 0        & 0         & 0 \\
    %         \vdots       & \vdots        & \vdots         & \vdots          & \ddots & \vdots   & \vdots    & \vdots \\
    %         0            & 0             & 0              & 0               & \cdots & -1  & 2      & -1 \\
    %         0            & 0             & 0              & 0               & \cdots & 0        & -1      & 1 
    %     \end{pmatrix}
    % \end{equation}


    % We could see easily see that both $A$ and $H^{-1}A$ are {symmetric}. Now, let  $D = H^{-1}Q$.   %$D = H^{-1}Q$. 
    % In the next section, we will prove that our operator $D$ satisfy SBP property from Definition \ref{def-sbp-operator}. 
    % Now, we're closing this section with implementing the boundary condition with penalty parameter $\tau$ that goes as follows. 
    % \begin{equation}\label{semidiscrete}
    %     \frac{d\bar{\mathbf{q}}}{dt} + D\mathbf{f}(\bar{\mathbf{q}}) 
    %         + \alpha H^{-1}A\bar{\mathbf{q}} + \tau H^{-1} \mathbf{e}_0(\bar{{q}}_0  - g(t)) = 0
    % \end{equation}
    % The last terms is used since the boundary condition imposed at the left boundary. that's why we multiply the condition $(\bar{{q}}_0  - g(t))$
    % with  $H^{-1} \mathbf{e}_0$. 
    
    Introducing some notation $\mathbf{Q}, \mathbf{A},$ and $\mathbf{H}$ as follows:
    \begin{align}
        \mathbf{Q} = \begin{pmatrix}
                - \frac12         & \frac12            & 0              & 0               & \cdots & 0        & 0         & 0 \\
                -\frac{1}{2} & 0             & \frac{1}{2}    & 0               & \cdots & 0        & 0         & 0 \\
                0            & - \frac{1}{2} & 0              & \frac{1}{2}     & \cdots & 0        & 0         & 0 \\
                \vdots       & \vdots        & \vdots         & \vdots          & \ddots & \vdots   & \vdots    & \vdots \\
                0            & 0             & 0              & 0               & \cdots & -\frac{1}{2} & 0      & \frac{1}{2} \\
                0            & 0             & 0              & 0               & \cdots & 0        & -\frac12      & \frac12
        \end{pmatrix} \\[6pt]
        \mathbf{A} = \begin{pmatrix}
            {1}           &{-1}            & 0              & 0               & \cdots & 0        & 0         & 0 \\
            -1           & 2             & -1    & 0               & \cdots & 0        & 0         & 0 \\
            0            & -1            & 2              & -1      & \cdots & 0        & 0         & 0 \\
            \vdots       & \vdots        & \vdots         & \vdots          & \ddots & \vdots   & \vdots    & \vdots \\
            0            & 0             & 0              & 0               & \cdots & -1  & 2      & -1 \\
            0            & 0             & 0              & 0               & \cdots & 0        & -1      & 1 
        \end{pmatrix}
    \end{align}
    and $\mathbf{H} =  \text{diag} (\frac{1}{2}\Delta x_0,\Delta x_1,\Delta x_2,\cdots,\Delta x_{N-1},\frac{1}{2}\Delta x_N)$. 
    Also notating the vector $\mathbf{q} = (q_0, q_1, \dots, q_N)^\top $.
    We could write our equation as 
    \begin{align}
        \mathbf{H}\frac{d\bar{\mathbf{q}}}{dt} + \mathbf{Q}\mathbf{f}(\bar{\mathbf{q}}) 
            + \alpha \mathbf{A}\bar{\mathbf{q}} &= 0
    \end{align}
    or if we write $\mathbf{D}=\mathbf{H}^{-1}\mathbf{Q}$
    \begin{align}
        \frac{d\bar{\mathbf{q}}}{dt} + \mathbf{D}\mathbf{f}(\bar{\mathbf{q}}) 
            + \alpha \mathbf{H}^{-1}\mathbf{A}\bar{\mathbf{q}} &= 0
    \end{align}
    

    We could see easily see that both $\mathbf{A}$ and $\mathbf{H}^{-1}\mathbf{A}$ are {symmetric}. 
    Let the differential operator in the discrete level estimated by our operator $\mathbf{D}$ .
    \begin{align}
        \label{SBP-operator-FV}
        \frac{d\mathbf{q}}{dx}  \approx \mathbf{D}_x \mathbf{q}
    \end{align}
     
    In the next section, we will prove that our operator $\mathbf{D}$ satisfy SBP property from Definition \ref{def-sbp-operator}. 
    Now, we're closing this section with implementing the boundary condition with penalty parameter $\tau$ that goes as follows. 
    \begin{equation}\label{semidiscrete}
        \frac{d\bar{\mathbf{q}}}{dt} + \mathbf{D}\mathbf{f}(\bar{\mathbf{q}}) 
            + \alpha \mathbf{H}^{-1}\mathbf{A}\bar{\mathbf{q}} + \tau \mathbf{H}^{-1} \mathbf{e}_0(\bar{{q}}_0  - g(t)) = 0
    \end{equation}
    The last terms is used since the boundary condition imposed at the left boundary. that's why we multiply the condition $(\bar{{q}}_0  - g(t))$
    with  $H^{-1} \mathbf{e}_0$. 
    \subsection{Stability}
    To ensure the stability, we will check it with energy methods and adjust 
    the penalty parameter $\tau$. 
    Now, we well introduce a discrete inner product that related with the operator from
    previous subsection. 
    \begin{definition}
        The inner product $\left<\cdot,\cdot\right>_H$ defined by
        \[
            \left\langle \mathbf{p},\mathbf{q}\right\rangle_H = \mathbf{q}^\top H\mathbf{p}
        \]
        with $H=\text{diag}(\frac12,1,1,\dots,1,\frac12)$. The norm based
        in the inner product defined by 
        \[
            \left\| \mathbf{q} \right\|^2 = \left\langle \mathbf{q},\mathbf{q}\right\rangle_H. 
        \]
    \end{definition}
    We can easily prove that this is actually an inner product. Moreover, with this 
    inner product, our operator $D$ becomes an SBP operator since 
    \begin{align}
        \left\langle \mathbf{D}\mathbf{p},\mathbf{q}\right\rangle 
        + \left\langle \mathbf{p},\mathbf{D}\mathbf{q}\right\rangle 
        &= \mathbf{q}^\top \mathbf{H}(\mathbf{D}\mathbf{p}) + (\mathbf{D}\mathbf{q})^\top \mathbf{H}\mathbf{p} \\
        &= \mathbf{q}^\top \mathbf{H}(\mathbf{H}^{-1}\mathbf{Q})\mathbf{p} + \mathbf{q}^\top (\mathbf{H}^{-1}\mathbf{Q})^\top \mathbf{H}\mathbf{p}\\
        &= \mathbf{q}^\top \mathbf{Q}\mathbf{p} + \mathbf{q}^\top \mathbf{Q}^\top \mathbf{p} \\
        &= \mathbf{q}^\top (\mathbf{Q}+\mathbf{Q}^\top )\mathbf{p} \\
        &= \mathbf{q}^\top (\mathbf{e}_N\mathbf{e}_N^\top  - \mathbf{e}_0\mathbf{e}_0^\top )\mathbf{p}\\
        % &= \mathbf{q}_(N)\mathbf{p}(N) + \mathbf{q}(0)\mathbf{p}(0)
        &= \mathbf{q}_N \mathbf{p}_N  - \mathbf{q}_0 \mathbf{p}_0  
    \end{align}
    Hence, by Definition (\ref{def-sbp-operator}), $D$ is an SBP operator. 
    
    
    Now, moving to our expression (\ref{semidiscrete}), we will make sure that 
    our numerical solution is stable. With abuse of notation, multiply expression (\ref{semidiscrete}) 
    with $\bar{\mathbf{q}}$ from the left, we get 
    \begin{align}
        \left\langle \bar{\mathbf{q}}, \frac{d\bar{\mathbf{q}}}{dt} 
                + \mathbf{D}{\bar{\mathbf{q}}} + \alpha \mathbf{H}^{-1}A{\bar{\mathbf{q}}} 
                + \tau \mathbf{H}^{-1} \mathbf{e}_0(A{\bar{\mathbf{q}}}  - g(t))\right\rangle_\mathbf{H} &= 0 \\
        \left\langle \bar{\mathbf{q}}, \frac{d\bar{\mathbf{q}}}{dt} \right\rangle_\mathbf{H} 
                + \left\langle \bar{\mathbf{q}},  \mathbf{D}{\bar{\mathbf{q}}} \right\rangle_\mathbf{H} 
                +\left\langle \bar{\mathbf{q}}, \alpha \mathbf{H}^{-1}\mathbf{A}{\bar{\mathbf{q}}} \right\rangle_\mathbf{H}
                +\left\langle \bar{\mathbf{q}}, \tau \mathbf{H}^{-1} \mathbf{e}_0(\bar{{q}}_0  - g(t))\right\rangle_\mathbf{H} &=0 \\
        \frac12 \frac{d}{dt}\left\langle \bar{\mathbf{q}}, \bar{\mathbf{q}} \right\rangle_\mathbf{H} 
                + \bar{\mathbf{q}}^\top \mathbf{D}^\top \mathbf{H}\bar{\mathbf{q}}
                + \alpha(\mathbf{H}^{-1}\mathbf{A}\bar{\mathbf{q}})^\top \mathbf{H}\bar{\mathbf{q}}
                +\tau (\mathbf{H}^{-1} \mathbf{e}_0(\bar{{q}}_0  - g(t)))^\top \mathbf{H}\bar{\mathbf{q}} &=0 \\
        \frac12\frac{d}{dt} \| \bar{\mathbf{q}}  \|^2 
                + \bar{\mathbf{q}}^\top \mathbf{Q}^\top \bar{\mathbf{q}} 
                + \alpha \bar{\mathbf{q}}^\top \mathbf{A}^\top \bar{\mathbf{q}} + \tau \bar{{q}}_0(\bar{{q}}_0 - g(t)) &=0 
    \end{align}
    The third row to fourth row use the following identity, $\mathbf{D}=\mathbf{H}^{-1}\mathbf{Q}$. 
    If we multiply the equation from the right, with the same way we will get
    \[
        \frac12 \frac{d}{dt} \| \bar{\mathbf{q}}  \|^2 
            + \bar{\mathbf{q}}^\top \mathbf{Q}\bar{\mathbf{q}} 
            + \alpha \bar{\mathbf{q}}^\top \mathbf{A}\bar{\mathbf{q}} + \tau \bar{{q}}_0(\bar{{q}}_0 - g(t)) =0 
    \] 
    Adding both of them and remember that $\mathbf{A} = \mathbf{A}^\top $, we could get
    \begin{align}
        \frac{d}{dt} \| \bar{\mathbf{q}}  \|^2 
                + \bar{\mathbf{q}}^\top (\mathbf{Q}^\top  + \mathbf{Q}) \bar{\mathbf{q}} 
                + 2\alpha \bar{\mathbf{q}}^\top \mathbf{A}^\top \bar{\mathbf{q}} + 2\tau \bar{{q}}_0(\bar{{q}}_0 - g(t)) &=0 \\
        \frac{d}{dt} \| \bar{\mathbf{q}}  \|^2 + (\bar{{q}}_N^2 - \bar{{q}}_0^2) + 2\alpha \bar{\mathbf{q}}^\top \mathbf{A}^\top \bar{\mathbf{q}} + 2\tau \bar{{q}}_0(\bar{{q}}_0 - g(t)) &=0 \label{QtplusQ} \\ 
        \frac{d}{dt} \| \bar{\mathbf{q}}  \|^2 + 2\alpha \bar{\mathbf{q}}^\top \mathbf{A}^\top \bar{\mathbf{q}} 
            + \bar{{q}}_N^2 + (2\tau -1) \bar{{q}}_0^2 - 2\tau \bar{{q}}_0 g(t) &=0 \\
        \frac{d}{dt} \| \bar{\mathbf{q}}  \|^2 + 2\alpha \bar{\mathbf{q}}^\top \mathbf{A}^\top \bar{\mathbf{q}} + \bar{{q}}_N^2 
            + (2\tau -1)\left( \bar{{q}}_0^2 - 2\frac{\tau}{2\tau -1} \bar{{q}}_0 g(t) \right) &=0 \\
            \frac{d}{dt} \| \bar{\mathbf{q}}  \|^2 + 2\alpha \bar{\mathbf{q}}^\top \mathbf{A}^\top \bar{\mathbf{q}} + \bar{{q}}_N^2 
            + (2\tau -1)\left( \bar{{q}}_0^2 - 2\bar{{q}}_0 \frac{\tau g(t)}{2\tau -1} + \left(\frac{\tau g(t)}{2\tau -1}\right)^2  
                - \left(\frac{\tau g(t)}{2\tau -1}\right)^2 \right)&=0 \\
            \frac{d}{dt} \| \bar{\mathbf{q}}  \|^2 + 2\alpha \bar{\mathbf{q}}^\top \mathbf{A}^\top \bar{\mathbf{q}} + \bar{{q}}_N^2 
                + (2\tau -1)\left( \bar{{q}}_0^2 -  \left(\frac{\tau g(t)}{2\tau -1}\right) \right)^2 
                    + \left(- \frac{\tau^2 g(t)^2}{2\tau -1} \right) &=0 \label{gettau}
        \end{align}
    The equation (\ref{QtplusQ}) came from the fact that $\mathbf{Q}+\mathbf{Q}^\top $ entries all zeros except the first entry
    of the diagonal and the last entry of the diagonal. Remember that $A$ is a positive semi-definite matrix. 
    If we could bound $2\tau -1 >0$, then $-\frac{1}{2\tau -1}<0$ and both last two terms in (\ref{gettau}) are positive. 
    That means 
    \begin{align}
        \frac{d}{dt} \| \bar{\mathbf{q}}  \|^2 \leq - 2\alpha \bar{\mathbf{q}}^\top \mathbf{A}^\top \bar{\mathbf{q}} - \bar{{q}}_N^2  
            - (2\tau -1)\left( \bar{{q}}_0^2 -  \left(\frac{\tau g(t)}{2\tau -1}\right) \right)^2 
                + \frac{\tau^2 g(t)^2}{2\tau -1} \leq  \frac{\tau^2 g(t)^2}{2\tau -1}. 
    \end{align}
    or if we had $\tau > \frac{1}{2}$ we could safely say that our numerical solution $\bar{\mathbf{q}}$ is stable.

\section{Numerical Experiment}
    \newpage
    
\section{Shallow Water Equation}
    In this section we want to use SBP-SAT with finite volume 
    to model the linearized one dimensional shallow water equation 
    using the same method as the advection equation in the previous section.
\subsection{Linearization}
    Let $h$ be depth of the water and $u$ be $x$-directional velocity. 
    The one dimensional shallow water equation is the following system of equation.  
    \begin{align}
        \frac{\partial h}{\partial t} &+ \frac{\partial uh}{\partial x} = 0 \\
        \frac{\partial (uh)}{\partial t} &+ \frac{\partial (u^2 h + \half gh^2) }{\partial x} = 0
    \end{align}
    We'll linearized this equation by letting $h = H + h'$ and $u = U + u'$ 
    for some constant $H$ and $U$ and small perturbation $h'$ and $u'$. 

    Now, putting it into the system that we have, we got the following,
    \begin{align}
        \frac{\partial}{\partial t} (H+h') + \frac{\partial}{\partial x}\left((U+u')(H+h')\right) = 0\\
        \frac{\partial}{\partial t} \left((U+u')(H+h')\right) 
                + \frac{\partial}{\partial x} \left((U+u')^2(H+h')
                        +\half g (H+h')    \right)=0
    \end{align}
    Note that $H$ and $U$ are constants, we got
    \begin{align}
        \frac{\partial h'}{\partial t} 
            &+ H \frac{\partial u'}{\partial x} 
            + U \frac{\partial h'}{\partial x} 
            + \frac{\partial u'h'}{\partial x} = 0 \\
        H \frac{\partial u'}{\partial t} 
            + U \frac{\partial h'}{\partial t} 
            + \frac{\partial u'h'}{\partial t}
               &+ 2UH \frac{\partial u'}{\partial x} 
                +   H \frac{\partial u'^2}{\partial x} 
                +  U^2\frac{\partial h'}{\partial x} 
                + 2U  \frac{\partial u'h'}{\partial x} 
                + \frac{\partial u'^2h}{\partial x} +  \notag                     \\
               &\  \frac{\partial \left(\half gH^2\right)}{\partial x} 
                + gH \frac{\partial h'}{\partial x} 
                + \half g \frac{\partial h'^2}{\partial x} =0
    \end{align}
    Ignoring the higher terms $\frac{\partial u'h'}{\partial t},
    \frac{\partial u'h'}{\partial x}, \frac{\partial u'^2}{\partial x},
    \frac{\partial u'^2h'}{\partial x}$, we get
    \begin{align}
        \label{sys63}
        &\frac{\partial h'}{\partial t} + H \frac{\partial u'}{\partial x} + U \frac{\partial h'}{\partial x} =0 \\
        \label{sys64}
        & H \frac{\partial u'}{\partial t} + U \frac{\partial h'}{\partial t} 
            + 2UH \frac{\partial u'}{\partial x} 
            +  U^2\frac{\partial h'}{\partial x}  
            + gH \frac{\partial h'}{\partial x} =0
    \end{align}
    Substituting (\ref{sys63}) into (\ref{sys64}) to eliminate $\frac{\partial h'}{\partial t}$, it becomes
    \begin{align}
        H \frac{\partial u'}{\partial t} 
            + U \left( - H \frac{\partial u'}{\partial x} - U \frac{\partial h'}{\partial x} \right)
            + 2UH \frac{\partial u'}{\partial x} 
            +  U^2\frac{\partial h'}{\partial x}  
            + gH \frac{\partial h'}{\partial x} =0
    \end{align} 
    or equivalently
    \begin{align}
        H \frac{\partial u'}{\partial t} + UH \frac{\partial u'}{\partial x}  + gH \frac{\partial h'}{\partial x} =0
    \end{align}
    Canceling out the constant $H$ from the equation, we got the following Shallow Water equation linearized system. 
    \begin{align}
        &\frac{\partial h'}{\partial t} + H \frac{\partial u'}{\partial x} + U \frac{\partial h'}{\partial x} =0 \\
        &\frac{\partial u'}{\partial t} + U \frac{\partial u'}{\partial x}  + g \frac{\partial h'}{\partial x} =0
    \end{align}
    or with change of notation, we could rewrite it to the following systems. 
    \begin{align}
        \label{swe1}
        &\frac{\partial h}{\partial t}  + U \frac{\partial h}{\partial x}+ H \frac{\partial u}{\partial x} =0 \\
        \label{swe2}
        &\frac{\partial u}{\partial t} + g \frac{\partial h}{\partial x} + U \frac{\partial u}{\partial x}  =0
    \end{align}
\subsection{Analysis}
    In this section we would like to analyse the stability with energy method
    to get our boundary condition. 

    Multiply equation (\ref{swe1}) with $gh$, and integrate with respect to $x$ in the inteval $[0,1]$,
    we got
    \begin{align}
        gh\frac{\partial h}{\partial t}+ gUh \frac{\partial h}{\partial x}  + gHh \frac{\partial u}{\partial x} =0\\
        \int_0^1 gh\frac{\partial h}{\partial t}+ gUh \frac{\partial h}{\partial x}  + gHh \frac{\partial u}{\partial x} + gUh \frac{\partial h}{\partial x} =0\\
        \int_0^1 gh\frac{\partial h}{\partial t} \dx  +  \int_0^1 gUh  \frac{\partial h}{\partial x} \dx + \int_0^1 gHh  \frac{\partial u}{\partial x} \dx=0\\
        \frac{d}{dt} \int_0^1 \half gh^2 \dx + gU \left( \half h^2 \big\vert_0^1 \right) 
            + gH \left( uh\big{\vert}_0^1 - \int_0^1 u \frac{\partial h}{\partial x} \dx \right) =0 \label{sys74}
    \end{align}
    Multiply equation ({\ref{swe2}}) with $uH$ and integrate with respect to $x$ in in the interval $[0,1]$, we got
    \begin{align}
        uH\frac{\partial u}{\partial t} + uHg \frac{\partial h}{\partial x} + uHU \frac{\partial u}{\partial x}  =0 \\
        \int_0^1 uH\frac{\partial u}{\partial t} + uHg \frac{\partial h}{\partial x} + uHU \frac{\partial u}{\partial x} \dx  =0 \\
        \int_0^1 uH\frac{\partial u}{\partial t} \dx + \int_0^1 uHg \frac{\partial h}{\partial x} \dx +  \int_0^1 uHU \frac{\partial u}{\partial x} \dx =0 \\
        \frac{d}{dt} \int_0^1 \half Hu^2 \dx + gH \int_0^1 u \frac{\partial h }{\partial x} \dx + HU \int_0^1 u\frac{\partial u}{\partial x} \dx \\
        \frac{d}{dt} \int_0^1 \half Hu^2 \dx + gH \int_0^1 u \frac{\partial h }{\partial x} \dx + HU \left( \half u^2 \big\vert_0^1 \right)  =0 \label{sys79}
    \end{align}
    Adding (\ref{sys74}) with (\ref{sys79}) will cancel out terms $gH \int_0^1 u \frac{\partial h }{\partial x} \dx$ and we got
    \begin{align}\label{sys80}
        \frac{d}{dt} \int_0^1 \half \left(gh^2 + Hu^2 \right) dx 
            + gU \left( \half h^2 \big\vert_0^1 \right)
            + gH \left( uh\big{\vert}_0^1 \right)
            + HU \left( \half u^2 \big\vert_0^1 \right) = 0
    \end{align}
    Here, we could see with energy method, that this system of equation is stable. 
    
    Now, we want to know how many boundary condition needed in this 
    linearized shallow water equation. 

    Let's rewrite equation (\ref{sys80}) by letting $E= \int_0^1 \half \left(gh^2 + Hu^2 \right) dx $ be the energy and 
    $\mathbf{q} = [h \quad u]^\top $. 
    The equation becomes
    \begin{align}
        \label{energyestimates-h-gH-u}
        \frac{dE}{dt} 
        + gU \left( \half h^2 \big\vert_0^1 \right)
        + gH \left( uh\big{\vert}_0^1 \right)
        + HU \left( \half u^2 \big\vert_0^1 \right) = 0
    \end{align}
    With some algebraic manipulation, we could get the equation equivalent to
    \begin{align}
        \frac{d{E}}{dt} + \left.
        \begin{bmatrix}h & u
        \end{bmatrix}
        \begin{bmatrix}
            \half gU & \half gH \\
            \half gH & \half Hu 
        \end{bmatrix}
        \begin{bmatrix}
            h \\ u
        \end{bmatrix}\right\vert_0^1 =0 
    \end{align}
    or 
    \begin{align} \label{qtAq_1}
        \frac{d{E}}{dt} + \left. \half
        \mathbf{q}^\top 
        \begin{bmatrix}
             gU &  gH \\
             gH &  HU 
        \end{bmatrix}\mathbf{q}\right\vert_0^1 = 0
    \end{align}
    We want to see the eigenvector of the symmetric matrix
    $M=\begin{bmatrix}gU&gH\\gH&HU\end{bmatrix}$ to get the boundary condition right. 
    Calculating $\text{det}(M-\lambda I)=0$ we get 
    \begin{align}
        \text{det}(M-\lambda I) = \left\vert 
            \begin{matrix}
            gU - \lambda & gH \\ gH & HU-\lambda
            \end{matrix}
        \right\vert  = 0 \notag \\
        (gU-\lambda)(HU - \lambda) - g^2H^2 = 0 \notag \\
        \lambda^2 - \lambda(gU+HU) + gHU^2 - g^2H^2 \notag \\
        \lambda_{1,2} = \half{(gU+HU) \pm \half \sqrt{(gU+HU)^2 - 4gH(U^2 - gH)}}
    \end{align}
    % Since the matrix $M$ is symmetric, we would have two real eigenvalue,
    % which means the first eigenvalue $\lambda_1 = \half{(gU+HU) + \half \sqrt{(gU+HU)^2 - 4gH(U^2 - gH)}}$ is positive real. 
    % Now, since we have the same terms $(gU+HU)$ inside the square root,  
    % we could see that the second second eigenvalue
    % $\lambda_2 > 0$ if and only if $ 4gH(U^2 - gH)>0$ or $U^2 > gH$. 
    % Hence, we have two boundary condition on the left if and only if $U^2 > gH$.
    % Equivalently, 
    % $\lambda_2 <0$ if and only if $U^2 < gH$.  Hence, we one boundary condition 
    % on the left and one boundary condition on the right. 

    We could rewrite the expression inside the root square as we need. 
    Two that useful would be 
    $$ (gU+HU)^2 - 4gH(U^2 - gH) = (2gH)^2 + U^2(g-H)^2 $$
    to make sure that the eigenvalue are reals, and the other one is 
    $$ (gU+HU)^2 - 4gH(U^2 - gH) = U^2(g+H)^2 + 4gH(gH - U^2)$$
    to see the sign of one of the eigenvalue. More on the second one, if we let $U>0$ and  
    $gH - U^2 > 0$, then we have the following: 
    \begin{align}
        \lambda_2 = \half U(g+H) - \half \sqrt{U^2(g+H)^2 + 4gH(gH - U^2)} %
        < \half U(g+H) - \half \sqrt{U^2(g+H)^2} < 0 
    \end{align}
    that means $\lambda_2 <0$. 
    Conversely, if we have $gH - U^2 < 0$, we will have $\lambda_2 >0$. 
    But this terms doesn't affect the first eigenvalue since both of
    the terms in the expression are positive so it would be always positive. 
    \[
        \half U(g+H) + \half \sqrt{U^2(g+H)^2 + 4gH(gH - U^2)}
    \]

    Of course, this would be change if we consider the other case, that is when $U<0$. 
    Our $\lambda_1$ would be positive or negative depends on the sign of $gh-U^2$ since
    $\half U(g+H)$ is negative. 
    \[
        \half U(g+H) + \half \sqrt{U^2(g+H)^2 + 4gH(gH - U^2)}
    \]
    If $gh-U^2 > 0$ then the expression is positive, otherwise it's negative. 
    Meanwhile, our $\lambda_2$ would always negative. 

    Now we have the following tables of sign
    \begin{table}[h!]
        \centering            
        % \begin{tabular}{|c|c||c|c|c|}
        %     \hline
        %     $U$ & $gH-U^2$ & $\lambda_1$  & $\lambda_2$ & \textbf{number of boundary condition}\\ \hline % \hhline{|=|=|=|=|}
        %     + & + & + & - & one on the left, one on the right\\ \hline
        %     + & - & + & + & two on the left \\ \hline
        %     - & + & + & - & one on the left, one on the right\\ \hline        
        %     - & - & - & - & two on the right \\ \hline        
        % \end{tabular}

        \begin{tabular}{|c|c||c|c|}
            \hline
            $U$ & $gH-U^2$ & $\lambda_1$  & $\lambda_2$ \\ \hline % \hhline{|=|=|=|=|}
            + & + & + & - \\ \hline
            + & - & + & + \\ \hline
            - & + & + & - \\ \hline        
            - & - & - & - \\ \hline        
        \end{tabular}

        % \caption[c]{Number of boundary based on eigenvalues}
        \caption[]{Eigenvalues sign}
        \label{eigen-boundary-1}
    \end{table}

    
    Now, lets back to the equation (\ref{qtAq_1}). For simplicity, we will take the
    factor $gH$ to get 

    \begin{align} \label{qtAq_2}
        \frac{d{E}}{dt} + \left. \half gH \,
        \mathbf{q}^\top 
        \begin{bmatrix}
             U/H &  1 \\
             1 &  U/g 
        \end{bmatrix}\mathbf{q}\right\vert_0^1 = 0
    \end{align}

    Addressing the factor, using the previous eigen analysis, we could see that this
    new matrices \begin{equation}\label{a_prime} A^\prime = \begin{bmatrix}
        U/H &  1 \\
        1 &  U/g 
   \end{bmatrix}\end{equation}  
   has eigenvalues $\frac1{2gH} \left(U(g+H) \pm \sqrt{U^2(g+H)^2 + 4gH(gH - U^2)}\right)$
   and would still have the same table of sign as Table \ref{eigen-boundary-1}. 
   
   Now, we would like to diagonalize matrix $ A^\prime $ from (\ref{a_prime}) to get the boundary treatments. 
   With the identity that we get from polynomial characteristic of matrix $ A^\prime $, 
   that is $\lambda_1 + \lambda_2 =\frac Ug + \frac UH$, one could find that the vector 
   \begin{equation}
        \begin{pmatrix}\lambda_1 - \frac Ug \\ 1 \end{pmatrix} \text{ and } 
        \begin{pmatrix}
            \lambda_2 - \frac Ug \\
            1
        \end{pmatrix}
    \end{equation}
    are the eigenvectors of $\lambda_1$ and $\lambda_2$ respectively. \
    Since the matrix $ A^\prime $ are symmetric, this eigenvectors should 
    be orthogonal to each other. But it also could came from the fact that 
    \begin{align}
        \left(\lambda_1 - \frac Ug\right)\left(\lambda_2 - \frac{U}{g}\right) + 1 
        &= \lambda_1 \lambda_2 - \left(\lambda_1 + \lambda_2\right) \frac{U}{g} + \frac{U^2}{g^2} + 1\\
        &= \left(\frac{U}{g}\frac{U}{H} -1 \right) - \left(\frac{U}{g} + \frac{U}{H}\right) \frac{U}{g}
                + \frac{U^2}{g^2} + 1 \\
        &= \left( \frac{U^2}{gH}  - 1 \right) 
            - \left( \frac{U^2}{g^2} + \frac{U^2}{gH} \right)+ \frac{U^2}{g^2} + 1 \\
        &= 0.
    \end{align}
    The fact that $\lambda_1 \lambda_2 = \left(\frac{U}{g}\frac{U}{H} -1 \right)$ also 
    came from the characteristic polynomial of $A^\prime$. 

    Now, since the the eigenvectors are orthogonal, hence we could construct 
    an orhtonormal matrix that diagonalize our matrix $A^\prime$ by normalizing the eigenvectors.
    Let $S$ be the matrix we mentioned, we have 
    \begin{equation}
        S = \begin{bmatrix}
            \frac 1c \left( \lambda_1 - \frac{U}{g}\right) & \frac 1d \left( \lambda_2 - \frac{U}{g}\right) \\
            \frac 1c & \frac{1}{d}
        \end{bmatrix}
    \end{equation}
    with $c = \sqrt[]{ \left( \lambda_1 - \frac{U}{g}\right)^2 + 1}$ and 
         $d = \sqrt[]{\left( \lambda_2 - \frac{U}{g}\right)^2 + 1}$, that is the length
    of our eigenvectors. 

    Note that $S$ is orhtonormal, we have $S^{-1} = S^\top $. 
    Therefore, our equation (\ref{qtAq_2}) becomes  
    \begin{equation}
        \frac{d{E}}{dt} + \left. \half gH \,
        \mathbf{q}^\top S
        \begin{bmatrix}
             \lambda_1 &  0 \\
             0 &  \lambda_2
        \end{bmatrix}S^\top\mathbf{q}\right\vert_0^1 = 0
    \end{equation}
    or equivalently
    \begin{equation}
        \frac{d{E}}{dt} + \left. \half gH \,
        (S^\top\mathbf{q})^\top
        \begin{bmatrix}
            \lambda_1 &  0 \\
            0 &  \lambda_2
       \end{bmatrix}S^\top\mathbf{q}\right\vert_0^1 = 0
    \end{equation}
    Notate $\mathbf{w} = S^\top \mathbf{q}$ and let $\mathbf{w} = [w_1 \ w_2]^\top$, 
    we have 
    \begin{align}\label{eq:w-to-q}
        w_1 = \frac{1}{c}\left(\lambda_1 -\frac{ U}{g}\right) h + \frac{1}{c} u, \quad  w_2 = \frac{1}{d}\left(\lambda_2 -\frac{ U}{g}\right) h + \frac{1}{d} u
    \end{align}
     % 
    \begin{equation}\label{eq:q-to-w}
        h = \frac{1}{c}\left(\lambda_1 -\frac{ U}{g}\right) w_1 +  \frac{1}{d}\left(\lambda_2 -\frac{ U}{g}\right) w_2, \quad  u = \frac{1}{c} w_1 + \frac{1}{d} w_2
    \end{equation}
    so that 
    \begin{equation}
        \frac{d{E}}{dt} + \left. \half gH \,
        \mathbf{w}^\top
        \begin{bmatrix}
            \lambda_1 &  0 \\
            0 &  \lambda_2
       \end{bmatrix}\mathbf{w}\right\vert_0^1 = 0
    \end{equation}
    Introducing
    \begin{equation}
        \textbf{BT} = - 
        \left.\mathbf{w}^\top
        \begin{bmatrix}
            \lambda_1 &  0 \\
            0 &  \lambda_2
       \end{bmatrix}
        \mathbf{w}\right\vert_0^1
    \end{equation}
    and moving terms to the right hand side we got
    \begin{equation}
        \frac{d{E}}{dt} = \half gH \, \textbf{BT}
    \end{equation}
    
    We could only focusing our attention to $\textbf{BT}$.
    \begin{equation}\label{eq:bt_correct}
        \textbf{BT} = - 
        \left.\mathbf{w}^\top
        \begin{bmatrix}
            \lambda_1 &  0 \\
            0 &  \lambda_2
       \end{bmatrix}
        \mathbf{w}\right\vert_0^1 
        =\left.\mathbf{w}^\top
        \begin{bmatrix}
            \lambda_1 &  0 \\
            0 &  \lambda_2
       \end{bmatrix}
        \mathbf{w}\right\vert_0 
        - 
        \left.\mathbf{w}^\top
        \begin{bmatrix}
            \lambda_1 &  0 \\
            0 &  \lambda_2
       \end{bmatrix}
        \mathbf{w}\right\vert_1
    \end{equation}
    % \textcolor{red}{
    %     There is a sign error in \eqref{eq:bt} which is propagated down the analysis.  I think the correct expression is
    %     \begin{equation}\label{eq:bt_correct}
    %         \textbf{BT} = - 
    %         \left.\mathbf{w}^\top
    %         \begin{bmatrix}
    %             \lambda_1 &  0 \\
    %             0 &  \lambda_2
    %     \end{bmatrix}
    %         \mathbf{w}\right\vert_0^1 
    %         =\left.\mathbf{w}^\top
    %         \begin{bmatrix}
    %             \lambda_1 &  0 \\
    %             0 &  \lambda_2
    %     \end{bmatrix}
    %         \mathbf{w}\right\vert_0 
    %         - 
    %         \left.\mathbf{w}^\top
    %         \begin{bmatrix}
    %             \lambda_1 &  0 \\
    %             0 &  \lambda_2
    %     \end{bmatrix}
    %         \mathbf{w}\right\vert_1
    %     \end{equation}
    %     Please can swap the boundary terms at $x =0$ and $x = 1$.
    % }
    
    Using $\mathbf{w} = [w_1 \ w_2]^\top$, 
    we could get 
    \begin{equation}\label{bt-terms}
        \textbf{BT} 
        = \left.\lambda_1 w_1^2 + \lambda_2 w_2^2\right\vert_0
        - \left.(\lambda_1 w_1^2 + \lambda_2 w_2^2)\right\vert_1. 
    \end{equation}
    
    \subsubsection{First case: $\lambda_1>0$, $\lambda_2<0$}
        In this case we have $U>0$ and $gh - U^2 >0$. On the right boundary, $x=1$, 
        we have need to get these expression to be negative, that is
        \begin{align}
            -\left(\left.\lambda_1 w_1^2 + \lambda_2 w_2^2\right)\right\vert_1 < 0
        \end{align}
        or equivalently
        \begin{align}
            \left.\lambda_1 w_1^2 + \lambda_2 w_2^2\right\vert_1 >0
        \end{align}
        Let $w_2 = \gamma_N w_1$ so that the expression becomes
        \begin{align}
            \left.\lambda_1 w_1^2 + \lambda_2 w_2^2\right\vert_1 
            &= \left.\lambda_1 w_1^2 + \lambda_2 (\gamma_N w_1)^2\right\vert_1 \\
            &= \left. w_1^2 \left( \lambda_1  + \lambda_2 \gamma_N^2\right) \right\vert_1
        \end{align}
        That means, we need to get $\lambda_1 + \lambda_2  \gamma_N^2>0$ when $x=1$. 
        Equivalently, we need to choose $\gamma_N^2$ such that 
        \begin{equation}\label{boundarycondition_gammaN}
            \gamma_N^2 < -\frac{\lambda_1}{\lambda_2}
        \end{equation}
        That would be our right boundary condition. 

        Meanwhile, on the left boundary, $x=0$, we need to have
        \begin{align}
            \left.\lambda_1 w_1^2 + \lambda_2 w_2^2\right\vert_0 < 0
        \end{align}
        But we know that $\lambda_1$ is positive. Let $w_1 = \gamma_0 w_2$. 

        \begin{align}
            \left.\lambda_1 w_1^2 + \lambda_2 w_2^2\right\vert_0
            &= \left.\lambda_1  (\gamma_0 w_2)^2 + \lambda_2 w_2^2\right\vert_0 \\
            &= \left. w_2^2 \left( \lambda_1\gamma_0^2   + \lambda_2  \right) \right\vert_0
        \end{align}
        so we need to have $\left( \lambda_1\gamma_0^2   + \lambda_2  \right) <0 $ when $x=0$. 
        Hence, our choosing of should fulfill the following. 
        \begin{equation}\label{boundarycondition_gamma0}
            \gamma_0^2 < -\frac{\lambda_2}{\lambda_1}
        \end{equation}
        

        \subsubsection{Second case: $\lambda_1 >0, \lambda_2 >0$}
        The second case, when $U>0$ and $gh - U^2 < 0$, that is when both $\lambda_1$ and $\lambda_2$
        are positive, from (\ref{bt-terms}), we could rest assure on the right side, since 
        the terms  $\left.(-\lambda_1 w_1^2 - \lambda_2 w_2^2) \right\vert_1$ is always non-positive. 
        On the other side, $x=1$, is pretty harsh since we need the expression
        $\left.(\lambda_1 w_1^2 + \lambda_2 w_2^2) \right\vert_0$
        to be non-positive. That would only possible if we force $w_1 = 0$ and $w_2 =0$, 
        and that would be two boundary conditions on the right. 
        % adding the implication for $h$ and $u$. 

        \subsubsection{Third case: $\lambda_1 <0, \lambda_2 >0$}
        % In this case we have $U<0$ and $gh - U^2 > 0$. 
        From the boundary terms (\ref{bt-terms}), we need $\mathbf{BT}$ to be non positive.  
        \begin{equation}
            \textbf{BT} 
            = \left.\lambda_1 w_1^2 + \lambda_2 w_2^2\right\vert_0
            - \left.(\lambda_1 w_1^2 + \lambda_2 w_2^2)\right\vert_1. 
        \end{equation}
        On the left boundary $x=0$, we have $\lambda_1 <0$ so we need to control the terms 
        $\left.\lambda_2 w_2^2\right\vert_0$.
        Setting $w_2 = \gamma_0 w_1$ we could get
        \begin{align}
            \left.\lambda_1 w_1^2 + \lambda_2 w_2^2\right\vert_0 
                &= \left.\lambda_1 w_1^2 + \lambda_2 ( \gamma_0 w_1)^2\right\vert_0 \\
                &= w_1^2 (\left.\lambda_1 + \lambda_2  \gamma_0 ^2)\right\vert_0 
        \end{align}
        We want this terms to be  non negative so our choice of $\gamma_0^2$ must satisfy 
        \begin{align}
            \lambda_1 + \lambda_2  \gamma_0 ^2 \leq 0 \\
             \gamma_0^2 \leq - \frac{\lambda_1 }{\lambda_2}. \label{gamma_0_secondcase}
        \end{align}
        On the right boundary $x=1$, we have $\gamma_2 >0$ so we need to control $\lambda_1 w_1^2$.
        Setting $w_1 = \gamma_N w_2$, we got 
        \begin{align}
            \left.(\lambda_1 w_1^2 + \lambda_2 w_2^2)\right\vert_1 
            &= \left.(\lambda_1 (\gamma_N w_2)^2 + \lambda_2 w_2^2)\right\vert_1 \\
            &= \left. w_2^2(\lambda_1 \gamma_N^2 + \lambda_2 )\right\vert_1 
        \end{align}
        We want this terms to be non positive, so we need to have 
        \begin{align}
            \lambda_1 \gamma_N^2 + \lambda_2 \geq 0\\
            \gamma_N^2 \leq -\frac{\lambda_2}{\lambda_1} \label{gamma_N_secondcase}
        \end{align}
        The sign changed because $\lambda_1 <0$. Now we need to have our choice of $\gamma_N$ 
        fulfill the condition (\ref{gamma_N_secondcase}).
        % For the other two cases, both with condition $U<0$, could be analyze with the same
        % argument. The number of boundary condition will follow suit: two negative eigenvalues 
        % would need two boundary condition on the left; one positive and one negative eigenvalue
        % would need one boundary condition on the left and one on the right. Here is why. 

        \subsubsection{Fourth case: $\lambda_1 <0, \lambda_2 <0$}
        In this case our $\mathbf{BT}$ that need to be non positive,
        \begin{equation}
            \textbf{BT} 
            = \left.\lambda_1 w_1^2 + \lambda_2 w_2^2\right\vert_0
            - \left.(\lambda_1 w_1^2 + \lambda_2 w_2^2)\right\vert_1. 
        \end{equation}
        would give us not so much room to relax. The condition $\lambda_1 <0, \lambda_2 <0$ 
        will ensure the non-positivity on the left boundary, and forcing us to 
        have $w_1=0$ and $w_0=0$ on the right boundary. 
        


    


\subsection{Discretization}
    We would like to do discretization of the interval $[0,1]$ as the previous section. 
    Consider Finite Volume Method on identical grid on the interval $[0,1]$. 
    Let $N$ be a positive integer and $I_i$ be the $i$-th control volume for all $i=1,2,...,N.$.
    Define $\Delta x_i = x_{i+\half} - x_{i-\half}$ be the length of the $i$-th interval 
    and $x_{i-\frac12}$ and $x_{i+\frac12}$ are the left boundary and right boundary
    of the interval respectively. 

    Let's our differential operator estimated by SBP operator 
    \begin{equation}
        \frac{\partial \mathbf{q}}{\partial x} \approx \mathbf{D}_x \mathbf{q}
    \end{equation}
    $\mathbf{D}_x = \mathbf{P}^{-1}\mathbf{Q}$.
    and
    $\mathbf{Q+Q^\top} = \textrm{diag}[-1,0,0,\dots,0,1]$
    
    discretize norm be 
    \begin{equation}
        \left< u,v\right>_\mathbf{P} = v^\top \mathbf{P}u.
    \end{equation}

    Our shallow water equation (\ref{swe1}) and (\ref{swe2}) could be rewrite as follows.
    \begin{align}
        P \frac{dh}{dt} + U\mathbf{Q} h + H\mathbf{Q}u = 0\\
        P \frac{du}{dt} + g\mathbf{Q}h + U\mathbf{Q}u = 0
    \end{align}
    For now, we would not adding the dissipative terms. 
    Consider the first case when $\lambda_1 >0$ and $\lambda_2 <0$ 
    that we have one boundary condition on the left and one boundary condition on the right. 
    % w_2 = \gamma_N w_1    ;  x = 1
    % w_1 = \gamma_0 w_2     : x = 0

    Add the SAT terms, we could write the discretize equation as 
    \begin{align}
        \mathbf{P} \frac{dh}{dt} + U\mathbf{Q} h + H\mathbf{Q}u 
            + \tau_{01} e_0 \left(w_2 - \gamma_0 w_1 \right)
            + \tau_{N1} e_N \left(w_1 - \gamma_N w_2 \right)
            = 0\\
        \mathbf{P} \frac{du}{dt} + g\mathbf{Q}h + U\mathbf{Q}u 
        + \tau_{02} e_0 \left(w_2 - \gamma_0 w_1 \right)
        + \tau_{N2} e_N \left(w_1 - \gamma_N w_2 \right)
        = 0
    \end{align}
    Multiply the two equation with $gh^\top$ and $Hu^\top$ respectively we got
    \begin{align}\label{swe-sat1}
        gh^\top \mathbf{P} \frac{dh}{dt} + gUh^\top \mathbf{Q} h + gHh^\top \mathbf{Q}u 
            + \tau_{01} gh_0 \left(w_2 - \gamma_0 w_1 \right)
            + \tau_{N1} gh_N \left(w_1 - \gamma_N w_2 \right)
            = 0\\ \label{swe-sat2}
        Hu^\top \mathbf{P} \frac{du}{dt} + gHu^\top \mathbf{Q}h + UHu^\top \mathbf{Q}u 
        + \tau_{02} Hu_0 \left(w_2 - \gamma_0 w_1 \right)
        + \tau_{N2} Hu_N \left(w_1 - \gamma_N w_2 \right)
        = 0
    \end{align}
    Transpose both of the two, we will get 
    \begin{align} \label{swe-sat3}
        g\left(\frac{dh}{dt}\right)^\top \mathbf{P}^\top h  
            + gU h^\top  \mathbf{Q}^\top h + gHu^\top \mathbf{Q}^\top h 
                + \tau_{01} gh_0 \left(w_2 - \gamma_0 w_1 \right)
                + \tau_{N1} gh_N \left(w_1 - \gamma_N w_2 \right)
            = 0\\ \label{swe-sat4}
        H \left(\frac{du}{dt}\right)^\top \mathbf{P}^\top u  
            + gHh^\top \mathbf{Q}^\top u + UHu^\top \mathbf{Q}^\top u 
                + \tau_{02} Hu_0 \left(w_2 - \gamma_0 w_1 \right)
                + \tau_{N2} Hu_N \left(w_1 - \gamma_N w_2 \right)
            = 0.
    \end{align}
    Adding (\ref{swe-sat1}), (\ref{swe-sat2}), (\ref{swe-sat3}), and (\ref{swe-sat4}) together
    and using the identity     $\mathbf{Q+Q^\top} = \textrm{diag}[-1,0,0,\dots,0,1]$
    gives us 
    \begin{align}
        \frac{d}{dt}\left(gh^\top \mathbf{P}h + H u^\top \mathbf{P}u \right) 
            &+ gUh^\top \mathbf{B} h + UHu^\top \mathbf{B} u 
            + gH \left(h^\top \mathbf{B}u + u^\top \mathbf{B}h \right) \nonumber \\
            &+ \tau_{01} gh_0 \left(w_2 - \gamma_0 w_1 \right)
            + \tau_{N1} gh_N \left(w_1 - \gamma_N w_2 \right) \nonumber \\
            &+ \tau_{02} Hu_0 \left(w_2 - \gamma_0 w_1 \right)
            + \tau_{N2} Hu_N \left(w_1 - \gamma_N w_2 \right) \nonumber \\
            &=0 
    \end{align}
    We will treat the terms $\left(gh^\top \mathbf{P}h + H u^\top \mathbf{P}u \right)$ 
    as our energy. Move all the terms except the energy, and evaluate some of the 
    terms involving $u^\top \mathbf{Q}u$ and $h^\top \mathbf{Q}h$, we got 
    \begin{align}
        \frac{d}{dt}\left(gh^\top \mathbf{P}h + H u^\top \mathbf{P}u \right) &=
        - gUh^\top \mathbf{B} h - UHu^\top \mathbf{B} u 
        - gH \left(h^\top \mathbf{B}u + u^\top \mathbf{B}h \right)\nonumber \\
        & \qquad  - \tau_{01} gh_0 \left(w_2 - \gamma_0 w_1 \right)
        - \tau_{N1} gh_N \left(w_1 - \gamma_N w_2 \right) \nonumber \\
        & \qquad  - \tau_{02} Hu_0 \left(w_2 - \gamma_0 w_1 \right)
        - \tau_{N2} Hu_N \left(w_1 - \gamma_N w_2 \right)\\
        & = -gU (h_N^2 - h_0^2) - UH(u_N^2 - u_0^2) - 2gH(h_Nu_N - h_0u_0) \nonumber \\
        & \qquad  - \tau_{01} gh_0 \left(w_2 - \gamma_0 w_1 \right)
        - \tau_{N1} gh_N \left(w_1 - \gamma_N w_2 \right) \nonumber \\
        & \qquad  - \tau_{02} Hu_0 \left(w_2 - \gamma_0 w_1 \right)
        - \tau_{N2} Hu_N \left(w_1 - \gamma_N w_2 \right) \label{swe-sat5} 
    \end{align}
    We could write the last expression (\ref{swe-sat5}) in the matrix form, 
    treating left boundary and right boundary separately. Before that, 
    for convenience, we would our boundary conditoin implemented in 
    the form of $h$ and $u$, namely, $w_2 - \gamma_0 w_1 = \alpha_0 h_0 + \beta_0 u_0$ on the left boundary; 
    $w_1 - \gamma_N w_2 = \alpha_N h_N + \beta_N u_N$ on the right boundary. 
    
    %%%%%%%%%%%%%%
    % wrong, need to change this later
    %
    % That would lead to
    % \begin{align}
    %     \alpha_0 = \frac{1}{c}\left(1 - \gamma_0 \left(\lambda_1 - \frac{U}{g}\right)\right)  \\
    %     \beta_0 =\frac{1}{d}\left(1 - \gamma_0 \left(\lambda_2 - \frac{U}{g}\right)\right)  \\
    % \end{align}

    Now, expression (\ref{swe-sat5}) becomes
    \begin{align}
        & -gU (h_N^2 - h_0^2) - UH(u_N^2 - u_0^2) - 2gH(h_Nu_N - h_0u_0) \nonumber \\
        & \qquad  - \tau_{01} gh_0 \left(w_2 - \gamma_0 w_1 \right)
        - \tau_{N1} gh_N \left(w_1 - \gamma_N w_2 \right) \nonumber \\
        & \qquad  - \tau_{02} Hu_0 \left(w_2 - \gamma_0 w_1 \right)
        - \tau_{N2} Hu_N \left(w_1 - \gamma_N w_2 \right) \label{swe-sat6}  \\
        &= gUh_0^2 + UHu_0^2 + 2gHh_0 u_0 - \tau_{01}gh_0 (\alpha_0 h_0+ \beta_0 u_0) - \tau_{02}Hu_0 (\alpha_0 h_0+ \beta_0 u_0) \label{swe-sat-mat2}\\
        &\qquad  -gU h_N^2 - UH u_N^2 - 2gHh_N u_N - \tau_{N1}gh_N(\alpha_N h_N+ \beta_N u) - \tau_{N2}Hu_N (\alpha_N h_N+ \beta_N u_N) \label{swe-sat-mat1}
    \end{align}
    From (\ref{swe-sat-mat2}) we could express it as 
    \begin{align}
        gH
        \begin{pmatrix}
            h_0 \\ u_0
        \end{pmatrix}^\top 
        \begin{pmatrix}
        \frac{U}{H} - \frac{2\tau_{01}\alpha_0 }{H} & 1 - \frac{\tau_{01}\beta_0}{H} - \frac{\tau_{02}\alpha_0}{g} \\
        1 - \frac{\tau_{01}\beta}{H} - \frac{\tau_{02}\alpha_0}{g} & \frac{U}{g} - \frac{2\tau_{02}\beta_0}{g}
        \end{pmatrix}
        \begin{pmatrix}
            h_0 \\ u_0
        \end{pmatrix}
    \end{align}
    and from (\ref{swe-sat-mat1}) we could express it as 
    \begin{align}
        -gH
        \begin{pmatrix}
            h_N \\ u_N
        \end{pmatrix}^\top 
        \begin{pmatrix}
        \frac{U}{H} +\frac{2\tau_{N1}\alpha_N }{H} & 1 + \frac{\tau_{N1}\beta_N}{H} + \frac{\tau_{N2}\alpha_N}{g} \\
        1 + \frac{\tau_{N1}\beta_N}{H} + \frac{\tau_{N2}\alpha_N}{g} & \frac{U}{g} + \frac{2\tau_{N2}\beta_N}{g}
        \end{pmatrix}
        \begin{pmatrix}
            h_N \\ u_N
        \end{pmatrix}
    \end{align}

    The whole equation now becomes
    \begin{align}
        \frac{d\mathbf{E}}{dt}  &=     gH
        \begin{pmatrix}
            h_0 \\ u_0
        \end{pmatrix}^\top 
        \begin{pmatrix}
        \frac{U}{H} - \frac{2\alpha_0 \tau_{01}}{H} & 1 - \frac{\tau_{01}\beta_0}{H} - \frac{\alpha_0}{g} \\
        1 - \frac{\tau_{01}\beta}{H} - \frac{\alpha_0}{g} & \frac{U}{g} - \frac{2\beta_0\tau_{02}}{g}
        \end{pmatrix}
        \begin{pmatrix}
            h_0 \\ u_0
        \end{pmatrix} \nonumber \\
        & \qquad -gH
        \begin{pmatrix}
            h_N \\ u_N
        \end{pmatrix}^\top 
        \begin{pmatrix}
        \frac{U}{H} +\frac{2\alpha_N \tau_{N1}}{H} & 1 + \frac{\tau_{N1}\beta_N}{H} + \frac{\alpha_N}{g} \\
        1 + \frac{\tau_{N1}\beta_N}{H} + \frac{\alpha_N}{g} & \frac{U}{g} + \frac{2\beta_N\tau_{N2}}{g}
        \end{pmatrix}
        \begin{pmatrix}
            h_N \\ u_N
        \end{pmatrix}
        \label{swe-sat-mat3}
    \end{align}
    To make it stable, we need to choose our penalty parameter
    so that the right hand side of the equation becomes non positive.
    We could make this easier by choose the parameter so that the matrix involved 
    becomes diagonal. That is, for left boundary, 
    $1 - \frac{\tau_{01}\beta_0}{H} - \frac{\tau_{02}\alpha_0}{g} = 0$ and 
    for the right boundary $1 + \frac{\tau_{N1}\beta_0}{H} + \frac{\tau_{N2}\alpha_0}{g} = 0$. 
    One particular choice would be: 
    \begin{align}
        \frac{\tau_{01}\beta_0}{H} = \frac{\tau_{02}\alpha_0}{g} = \half
        \quad \text{ and } \quad 
        \frac{\tau_{N1}\beta_0}{H} = \frac{\tau_{N2}\alpha_0}{g} = -\half
    \end{align}
    or equivalently
    \begin{align}
        \tau_{01} =  \frac{H}{2\beta_0}, \qquad
        \tau_{02} =  \frac{g}{2\alpha_0}, \qquad
        \tau_{N1} = -\frac{H}{2\beta_0}, \qquad
        \tau_{N1} = -\frac{g}{2\alpha_0}
    \end{align}

    By those choice, equation (\ref{swe-sat-mat3}) could be rewrite as
    \begin{align}
        \frac{d\mathbf{E}}{dt}  &=     gH
        \begin{pmatrix}
            h_0 \\ u_0
        \end{pmatrix}^\top 
        \begin{pmatrix}
        \frac{U}{H} - \frac{\alpha_0}{\beta_0}& 0 \\
        0 & \frac{U}{g} - \frac{\beta_0}{\alpha_0}
        \end{pmatrix}
        \begin{pmatrix}
            h_0 \\ u_0
        \end{pmatrix} \nonumber \\
        & \qquad -gH
        \begin{pmatrix}
            h_N \\ u_N
        \end{pmatrix}^\top 
        \begin{pmatrix}
        \frac{U}{H} +\frac{\alpha_N}{\beta_N}& 0 \\
        0 & \frac{U}{g} + \frac{\alpha_N}{\beta_N}
        \end{pmatrix}
        \begin{pmatrix}
            h_N \\ u_N
        \end{pmatrix}
    \end{align}
    As we want to have the energy non increasing, 
    we want the right hand side non positive. 
    It is sufficient to have the eigenvalues 
    $\frac{U}{H} - \frac{\alpha_0}{\beta_0}$ and 
    $\frac{U}{g} - \frac{\beta_0}{\alpha_0}$ are non positive
    and the eigenvalues
    $\frac{U}{H} +\frac{\alpha_N}{\beta_N}$ and 
    $\frac{U}{g} + \frac{\alpha_N}{\beta_N}$
    are non negative. Equivalently, the following inequality must be fulfilled.

    \begin{align}
        \label{boundarycondition_tau0}        
        \frac{U}{H} \leq \frac{\alpha_0}{\beta_0} \qquad &\text{and} \qquad \frac{U}{g} \leq \frac{\beta_0}{\alpha_0} \\
        \label{boundarycondition_tauN}
        \frac{U}{H} +\frac{\alpha_N}{\beta_N} \geq 0 \qquad &\text{and} \qquad \frac{U}{g} + \frac{\alpha_N}{\beta_N} \geq 0
    \end{align}
    % prove it case by case

\subsection{Boundary Conditions}
    The boundary condition that we will try to implementing should fulfill
    % the conditions (\ref{boundarycondition_gamma0}), (\ref{boundarycondition_gammaN}) 
    % to satisfy the stability in the semidiscrete level, and also satisfy 
    (\ref{boundarycondition_tau0}) and (\ref{boundarycondition_tauN}) to satisfy the stability 
    in the semi discrete level with penalty parameter chosen as well as some condtion 
    (\ref{boundarycondition_gamma0}), (\ref{boundarycondition_gammaN}), 
    (\ref{gamma_0_secondcase}), and (\ref{gamma_N_secondcase}) depending on the case. 
    But we have to back and fort 
    from using the pair $w_1$ and $w_2$, to using the pair $u$ and $h$. That's why we recall 
    equation (\ref{eq:w-to-q}) and (\ref{eq:q-to-w}).
    \begin{align}
        &w_1 = \frac{1}{c}\left(\lambda_1 -\frac{ U}{g}\right) h + \frac{1}{c} u, \qquad\qquad\qquad\qquad  w_2 = \frac{1}{d}\left(\lambda_2 -\frac{ U}{g}\right) h + \frac{1}{d} u \\
        %
        &h = \frac{1}{c}\left(\lambda_1 -\frac{ U}{g}\right) w_1 +  \frac{1}{d}\left(\lambda_2 -\frac{ U}{g}\right) w_2, \quad\qquad  u = \frac{1}{c} w_1 + \frac{1}{d} w_2
    \end{align}


    Before we're going into details. We would like to show that we couldn't set 
    the left boundary condition with $u=0$ or $h=0$. It's because if we look into 
    equation (\ref{energyestimates-h-gH-u}), and let one of the $u=0$ or $h=0$ we would end up 
    with either 
    \begin{align}
        % \frac{dE}{dt} 
        % + gU \left( \half h^2 \big\vert_0^1 \right) = 0  \quad \text{or} \quad 
        % \frac{dE}{dt} 
        % + HU \left( \half u^2 \big\vert_0^1 \right) = 0
        \cdots
    \end{align}
    which both of them are not energy stable. 


    \subsubsection{First case: $\lambda_1 >0, \lambda_2<0$.}
        In this case $\lambda_1 >0, \lambda_2<0$ would equivalent to 
        $U>0$ and $gH > U^2$. This case require one boundary condition in each side. 

        \subsubsection*{Right Boudary}
        Let our right boundary $h=0$. 
        Then, to have the stability in the semidiscrete level, we have to prove that this 
        satisfy (\ref{boundarycondition_gamma0}), (\ref{boundarycondition_gammaN}). 
        On the left, the condition $h=0$ give us  
        \begin{align}
            h &=0 \\
            \frac{1}{c}\left(\lambda_1 -\frac{ U}{g}\right) w_1 +  \frac{1}{d}\left(\lambda_2 -\frac{ U}{g}\right) w_2 & = 0 \\
            w_2 - \left(- \frac{\frac{1}{c}\left(\lambda_1 -\frac{ U}{g}\right)}{\frac{1}{d}\left(\lambda_2 -\frac{ U}{g}\right)} \right) w_1 &= 0 
        \end{align}
        So, we need our $\gamma_N^2 = \left(- \frac{\frac{1}{c}\left(\lambda_1 -\frac{ U}{g}\right)}{\frac{1}{d}\left(\lambda_2 -\frac{ U}{g}\right)} \right)^2$ 
        less than $-\frac{\lambda_1}{\lambda_2}$. 

        (Proof)

        Or we could also set the right boundary to be $u=0$. It will lead us to condition
        \begin{align}
            u &= 0 \\
            \frac{1}{c} w_1 + \frac{1}{d} w_2 & =0 \\
            w_2 - \left(- \frac{d}{c} \right) w_1 &= 0 \label{rightboundary:u}
        \end{align}
        and we need to prove that $\gamma_N^2 = \left(-\frac{d}{c}\right)^2 < -\frac{\lambda_1}{\lambda_2}$.

        (Proof)

        \subsubsection*{Left Boundary}
    In the left boundary, we need the linear combination of $u$ and $h$. 
    There's two sub-cases that interesting to be implemented. The first one 
    is when $\sqrt[]{\frac gH} h + u =0$ and the second one is $Hu + Uh = 0$ which is 
    the linearized version of $q = 0$. Remember that $q = hu$. In linearized version, $q = (H + h)(U+u) = HU + Hu + Uh + uh$. 
        Since we ignore the higher order terms, $uh$, and also $HU$ is only constant, we could 
        see that $q=0$ could be interpreted as $Hu + Uh = 0$.

    In the first sub-case, $\sqrt[]{\frac gH} h + u =0$, we could interpreted this as 
    \begin{align}
        \sqrt[]{\frac gH} h + u &=0 \\
        \sqrt[]{\frac gH}\left(\frac{1}{c}\left(\lambda_1 -\frac{ U}{g}\right) w_1 +  \frac{1}{d}\left(\lambda_2 -\frac{ U}{g}\right) w_2\right)
         + \left(\frac{1}{c} w_1 + \frac{1}{d} w_2 \right)&=0 \\
        \frac{1}{c}\left( \sqrt[]{\frac gH}  \left(\lambda_1 -\frac{ U}{g}\right) \right) w_1 
        + \frac{1}{d}\left( \sqrt[]{\frac gH}  \left(\lambda_2 -\frac{ U}{g}\right) \right) w_2 
        &= 0 \\
        w_1 - \left(-  \frac{
            \frac{1}{d}\left( \sqrt[]{\frac gH}  \left(\lambda_2 -\frac{ U}{g}\right) \right)
        }
        {
            \frac{1}{c}\left( \sqrt[]{\frac gH}  \left(\lambda_1 -\frac{ U}{g}\right) \right)
        }
        \right) w_2 &=0
    \end{align}
    Now, we need to prove that $$\gamma_0^2 = \left( - \frac{
        \frac{1}{d}\left( \sqrt[]{\frac gH} \left(\lambda_2 -\frac{ U}{g}\right) \right)
    }
    {
        \frac{1}{c}\left( \sqrt[]{\frac gH} \left(\lambda_1 -\frac{ U}{g}\right) \right)
    }
    \right)^2 < - \frac{\lambda_2}{\lambda_1}$$
    
    The other sub-case is $Hu + Uh = 0$. We could change this into 

    \begin{align}
        Hu + Uh &= 0\\
        H\left(
            \frac{1}{c}\left(\lambda_1 -\frac{ U}{g}\right) w_1 +  \frac{1}{d}\left(\lambda_2 -\frac{ U}{g}\right) w_2
            \right)
         + U\left(\frac{1}{c} w_1 + \frac{1}{d} w_2 \right)&= 0\\
        \left(\frac{H}{c}\left(\lambda_1 -\frac{ U}{g}\right) + \frac{U}{c}\right) w_1 
         + \left(\frac{H}{d}\left(\lambda_2 -\frac{ U}{g}\right) + \frac{U}{d}\right) w_2 &=0 \\
        w_1 - \left(- \frac{
            \frac{H}{c}\left(\lambda_1 -\frac{ U}{g}\right) + \frac{U}{c}
        }{
            \frac{H}{d}\left(\lambda_2 -\frac{ U}{g}\right) + \frac{U}{d}
        }\right) &= 0 
    \end{align}
    and as before, we need to prove that 
    $$
        \gamma_0^2 = \left(- \frac{
            \frac{H}{c}\left(\lambda_1 -\frac{ U}{g}\right) + \frac{U}{c}
        }{
            \frac{H}{d}\left(\lambda_2 -\frac{ U}{g}\right) + \frac{U}{d}
        }\right)^2 < - \frac{\lambda_2}{\lambda_1}.
    $$


    $$ \cdots $$ 

    In our implementation of SAT, we need to satisfy condition
    (\ref{boundarycondition_tau0}) and (\ref{boundarycondition_tauN}). 
    On the left boundary, it's pretty straight forward.
    In the first sub-case, $\sqrt[]{\frac gH} h + u =0$ 
    give us $\alpha_0 = \sqrt[]{\frac{g}{H}}$ and $\beta_0 = 1$. 
    We only need to show that 
    \begin{align}
        \frac{U}{H} &\leq \frac{\alpha_0}{\beta_0} 
        \qquad \text{and}  \qquad 
        \frac{U}{g} \leq \frac{\beta_0}{\alpha_0} \\
        \frac{U}{H} &\leq \sqrt[]{\frac{g}{H}}
        \qquad \text{and}  \qquad 
        \frac{U}{g} \leq \frac{1}{\sqrt[]{\frac{g}{H}}}
    \end{align}
    and both of the conditions equivalent in our first case $U^2 - gH < 0$. 

    The second sub-case, with $Hu + Uh = 0$, we have $\alpha_0 = U$ and $\beta_0 = H$. 
    And we need to prove that 
    \begin{align}
        \frac{U}{H} &\leq \frac{\alpha_0}{\beta_0} 
        \qquad \text{and}  \qquad 
        \frac{U}{g} \leq \frac{\beta_0}{\alpha_0} \\
        %
        \frac{U}{H} &\leq \frac{U}{H}
        \qquad \text{and}  \qquad 
        \frac{U}{g} \leq \frac{H}{U}
    \end{align}
    but thats equivalent to our condition $U^2 < gH$. 
    $$ \cdots $$ 



    \subsubsection{Second case: $\lambda_1 >0, \lambda_2>0$.}
        In this case, we have $U>0$ and $gH < U^2$. The two boundary conditions needed in this case 
        are $w_1 = 0$ and $w_2 = 0$. We don't need to choose $\gamma_0$ nor $\gamma_N$ in this case. 
        The only conditions we need to met are 
        (\ref{boundarycondition_tau0}) and (\ref{boundarycondition_tauN}). The boundary condition 
        $w_1 = 0$ would equivalent to 
        \begin{align}
            w_1 = \frac{1}{c}\left(\lambda_1 -\frac{ U}{g}\right) h_0 + \frac{1}{c} u_0 = \alpha_0 h_0 + \beta_0 u_0
        \end{align}
        so that 
        \begin{align}
            \frac{\alpha_0}{\beta_0} 
            = \frac{\frac{1}{c}\left(\lambda_1 -\frac{ U}{g}\right)}{\frac{1}{c}} 
            = \left(\lambda_1 -\frac{ U}{g}\right).
        \end{align}
        Hence, we need to prove that 
        \begin{align}
            \left(\lambda_1 -\frac{ U}{g}\right) \geq \frac{U}{H} 
            &\qquad \text{and} \qquad
            \frac{1}{\left(\lambda_1 -\frac{ U}{g}\right)} \geq \frac{U}{g} \\
            \lambda_1 \geq \frac{U^2}{gH} 
            &\qquad \text{and} \qquad
            \left(\lambda_1 -\frac{ U}{g}\right) \leq \frac{g}{U} 
        \end{align}
        We could see that this is true by looking this for any value of $U$ and $H$. 

        \[ DIAGRAM \]
        The second condition could be analyse with the same argument and get the following inequalities.
        \begin{align}
            \lambda_2 \geq \frac{U^2}{gH} 
            &\qquad \text{and} \qquad
            \left(\lambda_2 -\frac{ U}{g}\right) \leq \frac{g}{U} 
        \end{align}

        \[ DIAGRAM \]

    \subsubsection{Third case: $\lambda_1 <0, \lambda_2 >0$}
    In this case, we have $U<0$ and $gH > U^2$, one boundary condition on the left 
    and one boundary condition on the right. The condition (\ref{boundarycondition_tau0}) and (\ref{boundarycondition_tauN})
    do not depend on the cases of the eigenvalues, then we could reuse the result 
    from previous cases. So in this case we only need to check our choice 
    of $\gamma_0$ and $\gamma_N$. 
    \subsubsection*{Left Boundary}
    On the left boundary, we could only set the linear combination of $u$ and $h$. We would 
    look into two cases, $Uh + Hu =0$ and $\sqrt[]{\frac{g}{H}h}+u$. 

    \subsubsection*{Right Boundary}

    %% look again in the continuous level 
    %% do only one case for the linear combination of u and h

    
    \subsubsection{Fourth case: $\lambda_1 <0, \lambda_2 <0$}
    In this case, we have $U<0$ and $gH < U^2$, that is two boundary condition on the right. 
    Now we have plenty of choices: $u=0, h=0,$ and linear combination of $u$ and $h$. 
    Since the condition (\ref{boundarycondition_tau0}) and (\ref{boundarycondition_tauN})
    do not depend on the cases of the eigenvalues, then we could reuse the result 
    from previous cases. 
    Also because this case has $w_1 = 0$ and $w_2 =0$ we don't need to bother with 
    $\gamma_0$ and $\gamma_N$.


\section{Conclusion}



\end{document}



% Physical interpretation 
% as practical application

% numerical demosntration in the presentation
% with the transmisive boundary condtition 
% 




%% boundary condition is stable
%% try do things carefully